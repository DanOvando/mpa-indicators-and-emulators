% Options for packages loaded elsewhere
\PassOptionsToPackage{unicode}{hyperref}
\PassOptionsToPackage{hyphens}{url}
\PassOptionsToPackage{dvipsnames,svgnames,x11names}{xcolor}
%
\documentclass[
  default,
  lineno,
  referee]{sn-jnl}

\setlength\evensidemargin{\oddsidemargin}


\usepackage{amsmath,amssymb}
\usepackage{iftex}
\ifPDFTeX
  \usepackage[T1]{fontenc}
  \usepackage[utf8]{inputenc}
  \usepackage{textcomp} % provide euro and other symbols
\else % if luatex or xetex
  \usepackage{unicode-math}
  \defaultfontfeatures{Scale=MatchLowercase}
  \defaultfontfeatures[\rmfamily]{Ligatures=TeX,Scale=1}
\fi
\usepackage{lmodern}
\ifPDFTeX\else  
    % xetex/luatex font selection
\fi
% Use upquote if available, for straight quotes in verbatim environments
\IfFileExists{upquote.sty}{\usepackage{upquote}}{}
\IfFileExists{microtype.sty}{% use microtype if available
  \usepackage[]{microtype}
  \UseMicrotypeSet[protrusion]{basicmath} % disable protrusion for tt fonts
}{}
\makeatletter
\@ifundefined{KOMAClassName}{% if non-KOMA class
  \IfFileExists{parskip.sty}{%
    \usepackage{parskip}
  }{% else
    \setlength{\parindent}{0pt}
    \setlength{\parskip}{6pt plus 2pt minus 1pt}}
}{% if KOMA class
  \KOMAoptions{parskip=half}}
\makeatother
\usepackage{xcolor}
\setlength{\emergencystretch}{3em} % prevent overfull lines
\setcounter{secnumdepth}{-\maxdimen} % remove section numbering
% Make \paragraph and \subparagraph free-standing
\makeatletter
\ifx\paragraph\undefined\else
  \let\oldparagraph\paragraph
  \renewcommand{\paragraph}{
    \@ifstar
      \xxxParagraphStar
      \xxxParagraphNoStar
  }
  \newcommand{\xxxParagraphStar}[1]{\oldparagraph*{#1}\mbox{}}
  \newcommand{\xxxParagraphNoStar}[1]{\oldparagraph{#1}\mbox{}}
\fi
\ifx\subparagraph\undefined\else
  \let\oldsubparagraph\subparagraph
  \renewcommand{\subparagraph}{
    \@ifstar
      \xxxSubParagraphStar
      \xxxSubParagraphNoStar
  }
  \newcommand{\xxxSubParagraphStar}[1]{\oldsubparagraph*{#1}\mbox{}}
  \newcommand{\xxxSubParagraphNoStar}[1]{\oldsubparagraph{#1}\mbox{}}
\fi
\makeatother


\providecommand{\tightlist}{%
  \setlength{\itemsep}{0pt}\setlength{\parskip}{0pt}}\usepackage{longtable,booktabs,array}
\usepackage{calc} % for calculating minipage widths
% Correct order of tables after \paragraph or \subparagraph
\usepackage{etoolbox}
\makeatletter
\patchcmd\longtable{\par}{\if@noskipsec\mbox{}\fi\par}{}{}
\makeatother
% Allow footnotes in longtable head/foot
\IfFileExists{footnotehyper.sty}{\usepackage{footnotehyper}}{\usepackage{footnote}}
\makesavenoteenv{longtable}
\usepackage{graphicx}
\makeatletter
\def\maxwidth{\ifdim\Gin@nat@width>\linewidth\linewidth\else\Gin@nat@width\fi}
\def\maxheight{\ifdim\Gin@nat@height>\textheight\textheight\else\Gin@nat@height\fi}
\makeatother
% Scale images if necessary, so that they will not overflow the page
% margins by default, and it is still possible to overwrite the defaults
% using explicit options in \includegraphics[width, height, ...]{}
\setkeys{Gin}{width=\maxwidth,height=\maxheight,keepaspectratio}
% Set default figure placement to htbp
\makeatletter
\def\fps@figure{htbp}
\makeatother
% definitions for citeproc citations
\NewDocumentCommand\citeproctext{}{}
\NewDocumentCommand\citeproc{mm}{%
  \begingroup\def\citeproctext{#2}\cite{#1}\endgroup}
\makeatletter
 % allow citations to break across lines
 \let\@cite@ofmt\@firstofone
 % avoid brackets around text for \cite:
 \def\@biblabel#1{}
 \def\@cite#1#2{{#1\if@tempswa , #2\fi}}
\makeatother
\newlength{\cslhangindent}
\setlength{\cslhangindent}{1.5em}
\newlength{\csllabelwidth}
\setlength{\csllabelwidth}{3em}
\newenvironment{CSLReferences}[2] % #1 hanging-indent, #2 entry-spacing
 {\begin{list}{}{%
  \setlength{\itemindent}{0pt}
  \setlength{\leftmargin}{0pt}
  \setlength{\parsep}{0pt}
  % turn on hanging indent if param 1 is 1
  \ifodd #1
   \setlength{\leftmargin}{\cslhangindent}
   \setlength{\itemindent}{-1\cslhangindent}
  \fi
  % set entry spacing
  \setlength{\itemsep}{#2\baselineskip}}}
 {\end{list}}
\usepackage{calc}
\newcommand{\CSLBlock}[1]{\hfill\break\parbox[t]{\linewidth}{\strut\ignorespaces#1\strut}}
\newcommand{\CSLLeftMargin}[1]{\parbox[t]{\csllabelwidth}{\strut#1\strut}}
\newcommand{\CSLRightInline}[1]{\parbox[t]{\linewidth - \csllabelwidth}{\strut#1\strut}}
\newcommand{\CSLIndent}[1]{\hspace{\cslhangindent}#1}

%%%% Standard Packages

\usepackage{graphicx}%
\usepackage{multirow}%
\usepackage{amsmath,amssymb,amsfonts}%
\usepackage{amsthm}%
\usepackage{mathrsfs}%
\usepackage[title]{appendix}%
\usepackage{xcolor}%
\usepackage{textcomp}%
\usepackage{manyfoot}%
\usepackage{booktabs}%
\usepackage{algorithm}%
\usepackage{algorithmicx}%
\usepackage{algpseudocode}%
\usepackage{listings}%

%%%%

\raggedbottom
\makeatletter
\@ifpackageloaded{caption}{}{\usepackage{caption}}
\AtBeginDocument{%
\ifdefined\contentsname
  \renewcommand*\contentsname{Table of contents}
\else
  \newcommand\contentsname{Table of contents}
\fi
\ifdefined\listfigurename
  \renewcommand*\listfigurename{List of Figures}
\else
  \newcommand\listfigurename{List of Figures}
\fi
\ifdefined\listtablename
  \renewcommand*\listtablename{List of Tables}
\else
  \newcommand\listtablename{List of Tables}
\fi
\ifdefined\figurename
  \renewcommand*\figurename{Figure}
\else
  \newcommand\figurename{Figure}
\fi
\ifdefined\tablename
  \renewcommand*\tablename{Table}
\else
  \newcommand\tablename{Table}
\fi
}
\@ifpackageloaded{float}{}{\usepackage{float}}
\floatstyle{ruled}
\@ifundefined{c@chapter}{\newfloat{codelisting}{h}{lop}}{\newfloat{codelisting}{h}{lop}[chapter]}
\floatname{codelisting}{Listing}
\newcommand*\listoflistings{\listof{codelisting}{List of Listings}}
\makeatother
\makeatletter
\makeatother
\makeatletter
\@ifpackageloaded{caption}{}{\usepackage{caption}}
\@ifpackageloaded{subcaption}{}{\usepackage{subcaption}}
\makeatother
\ifLuaTeX
  \usepackage{selnolig}  % disable illegal ligatures
\fi
\usepackage{bookmark}

\IfFileExists{xurl.sty}{\usepackage{xurl}}{} % add URL line breaks if available
\urlstyle{same} % disable monospaced font for URLs
\hypersetup{
  pdftitle={Predicted Effects of Marine Protected Areas on Conservation and Catches are Highly Sensitive to Model Structure},
  pdfauthor={Daniel Ovando},
  pdfkeywords={MPA, Simulation Modeling, Model Selection, Conservation
Planning, Food Security},
  colorlinks=true,
  linkcolor={blue},
  filecolor={Maroon},
  citecolor={Blue},
  urlcolor={Blue},
  pdfcreator={LaTeX via pandoc}}

\title[Predicted Effects of Marine Protected Areas on Conservation and
Catches are Highly Sensitive to Model Structure]{Predicted Effects of
Marine Protected Areas on Conservation and Catches are Highly Sensitive
to Model Structure}

% author setup
\author*[1]{\fnm{Daniel} \sur{Ovando}}\equalcont{These authors contributed equally to this work.}
% affil setup
\affil[1]{\orgdiv{Ecosystem \& Bycatch Group}, \orgname{Inter-American
Tropical Tuna Commission}, \orgaddress{\street{8901 La Jolla Shores
Drive}, \city{La Jolla}, \postcode{92037}}}

% abstract 

\abstract{Use of Marine Protected Areas (MPAs) is expanding around the
world. While MPAs can have a wide variety of objectives, scientific
guidance on how to design MPAs to achieve objectives is generally based
on simulation modeling. While some types of model are not able to answer
some types of question, in other cases many different models may all
provide an answer of some kind, such as the predicted change in
population biomass and fisheries catches resulting from implementation
of an MPA. In these cases, the spatial and temporal scale of MPAs often
prevents us from determining which model is most parsimoneous
empirically, and as such the choice of model can be somewhat \emph{ad
hoc}. In this paper we compare the effects of MPAs on catch and biomass
predicted by a spatially explicit age-structured multi-species and
multi-fleet (\emph{High-definition}) model to the predictions generated
by a two-patch surplus production (\emph{Low-definition}) model fit to
emulate the \emph{High-definition} model. We found that while there are
some areas of agreement, in many cases the predictions of the two models
were markedly different, with the Low-definition model frequently
predicting substantially higher biomass and catch benefits from MPAs
than the High-definition model, and in many cases incorrectly estimating
the direction (positive or negative) of the MPA effects. Our results
show that care should be taken in selecting and interpreting the results
of MPA simultion models, and that research is needed to understand what
models are best suited to what policy recommendations when multiple
viable options exist.}

% keywords
\keywords{MPA,  Simulation Modeling,  Model Selection,  Conservation
Planning,  Food Security}

\begin{document}
\maketitle

\renewcommand*\contentsname{Table of contents}
{
\hypersetup{linkcolor=}
\setcounter{tocdepth}{3}
\tableofcontents
}
\subsection{Introduction}\label{introduction}

The field of ecology depends on models; it is how we move from simply
\emph{describing} ecological phenomena to \emph{understanding}
ecological processes. All models are abstractions of reality though,
requiring scientists to make decisions as to how best to approximate the
system in question to address the ecological question at hand. In some
cases, these decisions can be be based simply on first principles such
as our understanding of the basic physics of the universe. In many other
cases we are faced with multiple competing models of the ecological
process in question, with a common goal being the search for a
``parsimonious'' model that best explains the phenomena in question
without containing extraneous features that may reduce the performance
or efficiency of the model. For example, Murdoch et al. (2002) examined
under what conditions a single-species model may or may not be a valid
representation of a species existing in a multi-species
environment.While the scientific method has employed numerous practices
for judging the performance of competing models, for much of recent
history we have relied largely on confronting models with data (Hilborn
and Mangel 1997), whether through for example experimentation or
predictive skill commonly measured through various forms of information
criteria (Sutherland et al. 2023). Regardless of the method, the key
feature is that models are judged by their ability to explain observed
data in some ways shape or form.

More recently though, the increasing ``dimensionality'' of questions
asked by ecology (e.g.~number of species, spatial and temporal scales),
combined with the rapid increase in computation capabilities in the last
few decades has greatly expanded the use of a slightly different class
of model, which can generally be called ``simulation models''. The key
feature of these models is that while they can perhaps be ``tuned'' to
reality based on best available knowledge, they cannot be fit to data in
a conventional statistical sense, either due to a lack of degrees of
freedom (having many more parameters than available data), or an
inability to measure the phenomena being simulated on a reasonable time
scale.

These simulation models are extremely powerful, particularly as we
increasingly turn to ecological sciences to help make complex policy
decisions in a dynamic world. For example, the widely used Ecopath with
Ecosim (Christensen and Walters 2004) modeling framework allows for
complex simulation of the effects of environmental and/or policy drivers
on marine ecosystems that cannot realistically be directly evaluated
empirically. However, this power comes with limitations, namely in that
without the ability to rely solely on first principles or statistical
confrontation with data it is unclear how we should judge the
performance of alternative simulation models, particularly if
alternative models provide conflicting guidance as to the impact of a
phenomena or policy in question.

This challenge is well illustrated by the science of Marine Protected
Areas (MPAs). While MPAs have taken and continue to take numerous forms
throughout human history (Johannes 1978; Grorud-Colvert et al. 2021) ,
we refer to them here broadly as spatially defined areas in which some
forms of human activities are restricted. The most conventional form of
MPAs involve the restriction of some to all forms of extractive
activities such as fishing without their borders. The exact form of
these restrictions may vary but the choice of what is implemented is
generally made to achieve some policy objective, for example protection
of sensitive habitat, rebuilding of depleted fish populations, and
support of food security or economic objectives.

The use of MPAs to achieve various policy objectives has exploded in
recent years and looks to grow further, as best expressed by the goals
of the 30x30 movement, which seeks to place 30\% of terrestrial and
marine ecosystems in some form of protected area by 2030 (CBD 2022).
While individual MPAs may differ in their specific objectives, they
share a common theme that whatever the policy goal decisions must be
made as to how to design a given MPA to achieve a given set of
objectives, for example exactly where should the MPA be placed, how big
should it be, and exactly what rules will be implemented within its
borders.

The difficulty is that baring the most straight-forward of cases
(e.g.~protection of sensitive habitat of sessile species such as
deep-water corals), determining how to design MPAs to achieve a given
policy objective is almost entirely dependent on simulation models
(O'Leary et al. 2016). This is due in large part to the scale of the
processes involved in determining the outcomes of a given MPA or MPA
network. Many marine organisms move vast distances at one or more parts
of their life cycle (Leah R. Gerber et al. 2005). The impacts of
protection can take years or decades to develop (Nickols et al. 2019).
This means that empirical studies of the net effect of MPAs inside and
outside of their borders can require collection data over a large
spatial scale over many years, an important but costly task. Even once
those data are collected, observational studies lacking appropriate
statistical controls can easily misdiagnose the effects of a spatial
policy such as an MPA (Ovando et al. 2021; Ferraro, Sanchirico, and
Smith 2018).

Given then that we cannot realistically rely on experiments to design
large-scale networks of MPAs, and lack the vast time series of data
before and after MPAs at scale around the world to rely on observational
methods, we must turn to simulation modeling to determining how to
design MPAs that stand the best chance of achieving our policy
objectives. A vast array of simulation models have been developed to
fill this gap and help us project the potential policy impacts of MPAs
(Fulton et al. 2015). These models vary greatly in complexity, ranging
from single species two-patch biomass dynamics models with analytical
solutions (Hastings and Botsford 1999) to end-to-end spatially explicit
social-ecological models with thousands of parameters such as ATLANTIS
(Audzijonyte et al. 2019).

Several important works have surveyed the MPA modeling landscape. Leah
R. Gerber et al. (2003) reviewed a range of different MPA models and
summarized the types of processes included and common insights across
models. Pelletier and Mahévas (2005) evaluated a wide range of MPA
models and qualitatively scored their suitability for addressing
different kinds of policy questions, but found that a very wide range of
model complexities could be appropriate for broad questions like
addressing ``Enhancement of fisheries yield''. Fulton et al. (2015)
considered the landscape of models that have been used to model MPAs,
and provided recommendations on the types of models that might be
appropriate for asking different questions around MPAs. While this is an
extremely important contribution to the literature, the list of models
categorized as appropriate for particular MPA questions can span a vast
range, from single-species population dynamics models up to more
end-to-end style models such as
\href{https://osmose-model.org/publications/}{OSMOSE}. Across these
works then, while it is clear that some models are more capable than
others of addressing certain types of questions, conditional on being
capable of addressing a question (such as the impacts of MPAs on total
catch), it remains unclear to what extent the choice of a specific
viable model might influence the direction and magnitude of the effect
being studied.

The ideal case in this modelling landscape is that complexity is
additive in nature, meaning that all models capture the core processes
around MPAs in an unbiased manner, but some are better capable of
capturing finer-scale variation in MPA outcomes related to local
context. If this is the case, then we would expect the predictions of
say a two-patch biomass dynamics model to be right on average over a
region, providing strategic guidance as to the potential outcomes of
MPAs in that area, while a more detailed model is needed provide
tactical advice for the design of a specific closed area. While this
additive gradient in MPA model complexity would be ideal, we are unaware
of any work that has explicitly tested whether it exists. Answering this
question has real policy implications, particularly because models are
the primary tool used to provide advice on MPA design (O'Leary et al.
2016) (Fulton et al. 2015). If different models along a complexity
gradient do not simply provide more detailed illustrations of the same
image, but rather are painting completely different landscapes, the
predicted effects and subsequent design of MPA networks that will
increasingly affect marine ecosystems and dependent peoples may simply
be an artifact of the type of model selected rather than a reflection of
a robust understanding of the dynamics of these social-ecological
systems.

To illustrate the potential consequences of diverging predictions of
alternative models, consider the case of Sala et al. (2021). That paper
used a single species two-patch biomass dynamics modeling framework to
predict the conservation and food security outcomes of a proposed global
network of no-take MPAs, and as of publication of this paper is among
the most highly cited and widely publicized findings in the recent
marine conservation literature
(\url{https://nature.altmetric.com/details/101895056}).However, Ovando,
Liu, et al. (2023) showed that minor changes to core untestable
assumptions made by Sala et al. (2021) fundamentally altered both the
total predicted outcomes of MPAs for food security and the
prioritization map of this proposed global MPA network.

The question then is what models of MPAs are best suited for addressing
which policy questions when multiple different models can in theory
serve? The answer is mechanical in some cases. For example, if the goal
of a given policy is to increase the mean length of a given target
species, at minimum a model that tracks length is required. Conversely,
if the goal of a given MPA is to rebuild the trophic structure of a
fished ecosystem closer to unfished conditions, at minimum a
multi-species model is needed. How though do we determine what model to
use across a range of models that are all capable of tracking the same
outcomes? For example, two commonly desired outcomes of MPAs are changes
in total population biomass and total fishery catches. These metrics, or
at least representations of them, are produced both by a two-patch
biomass dynamics model and a full ecosystem model, and essentially every
MPA model in between. It is unclear then how to decide which of the many
models capable of simulation biomass and catch outcomes of MPAs is
parsimonious for the task at hand.

The question of exactly what model is appropriate for what question
related to spatial policies such as MPAs in marine social-ecological
systems if far beyond the scope of this paper. However, this paper seeks
to provide a step towards this broader goal by assessing whether models
of differing complexity provide complimentary or contradictory
predictions of the effects of MPAs across multiple policy dimensions.
Specifically, we simulated a range of MPA outcomes using a spatially
explicit age structured multi-species and multi-fleet simulation
framework presented in Ovando, Bradley, et al. (2023) called
\texttt{marlin}. We then tuned a two-patch biomass dynamics model
presented in Ovando, Liu, et al. (2023) and based on Cabral et al.
(2020), the foundation for the results presented in Sala et al. (2021),
to emulate the simulated dynamics of the more complex \texttt{marlin}
model. We compared the effects of no-take MPAs predicted by these two
models, and evaluated if and when the two models were in agreement.
While these are only two models out of the many available to model MPAs,
they are instructive as both types of models have been used to provide
predictions of the effects of MPAs on biomass and catch. We find
evidence that rather than existing along a gradient of strategic to
tactical, the two models provided vastly different and often
contradictory predictions of the outcomes of MPAs for food security and
conservation.

\subsection{Methods}\label{methods}

This paper uses two general classes of models that we call the
\emph{High} and \emph{Low} resolution models. The \emph{High-definition}
model uses a framework called \texttt{marlin} described in Ovando,
Bradley, et al. (2023) . The \texttt{marlin} model simulates arbitrary
numbers of independent age-structured fish-like populations distributed
in a two-dimensional seascape. These fish populations can be targeted by
arbitrary numbers of fishing fleets each of which can have individual
\emph{métiers} that specify the relationship between a given fleet and a
specific fish species. The \emph{Low-definition} model is two-patch
Pella-Tomlinson (Pella and Tomlinson 1969) surplus production model,
described in Ovando, Liu, et al. (2023) and based on Cabral et al.
(2020).

Each simulation follows the same general structure:

\begin{enumerate}
\def\labelenumi{\arabic{enumi}.}
\tightlist
\item
  Generate a set of state variables (e.g.~life history parameters and
  fleet dynamics) from a range of \emph{simple} (one species one fleet),
  \emph{medium} (four species one fleet), and \emph{complex} (four
  species two fleets) scenarios (Table~\ref{tbl-scenes})
\item
  Run the simulation with those state variables (Table~\ref{tbl-states})
  using the \emph{High-definition} model
\item
  Generate data from the \emph{High-definition} model and use it to
  estimate the parameters of the \emph{Low-definition} model
\item
  Apply a range of MPA designs comprised of size (percent of seascape)
  and placement strategy for the \emph{High-definition} case, and size
  only for the Low-definition model
\item
  Compare the percent change in biomass and catch predicted by the
  \emph{High} and \emph{Low-definition} models
\end{enumerate}

\subsubsection{High-definition Model and
Simulations}\label{high-definition-model-and-simulations}

The equations and processes describing \texttt{marlin} model used in the
\emph{High-definition} are presented in detail in Ovando, Bradley, et
al. (2023). Briefly, the model simulates fish populations using an
age-structured population. Length-at-age is calculated using the von
Bertalanffy growth equation with log-normally distributed variation in
the length at age. Natural mortality at age is calculated using the
length-inverse mortality function described in Lorenzen (2022).
Fecundity at age is a function of weight at age, with the potential for
hyperallometric fecundity in which fecundity increases faster than
weight. Recruitment follows a Beverton-Holt dynamics parameterized
around steepness (\emph{h}), with multiple options for the timing and
spatial scale of density dependence (e.g.~local vs.~global or pre-
vs.~post-dispersal). Movement is simulated using a continuous-time
Markov chain (CTMC), as described in Thorson et al. (2021). This process
breaks movement in a diffusion and taxis component, where the taxis
component allows fish to preferentially move towards better habitat. The
model runs in 6-month time steps for these simulations. Core life
history values are pulled from \texttt{Fishlife} (Thorson 2020).

Fishing fleets are simulated based on effort. Each fleet exerts an
amount of fishing effort in a time step, and chooses how to allocate
that effort in space based on a spatial-allocation strategy
(e.g.~revenue-per-unit-effort in the previous time step). Effort in a
given patch applies fishing mortality to simulated species based on the
\emph{métier} in question. The \emph{métier} specifies the contact
selectivity at age of that fleet for each simulated species, the price
per unit weight for each species, and the proportion of the total
fishing mortality experienced by that fish species contributed by the
\emph{métier} in question. Fishing fleets can have home ports, which add
a cost to fishing in a given patch that increase with linear distance
from the home port.

Together, these fish and fleet dynamics allow \texttt{marlin} to
simulate the effects of multiple fishing fleets on multiple species. For
example, by simulating multiple fleets and fish species simultaneously,
the model can capture bycatch dynamics in which a fleet targets one
species but incidentally captures another species if it occurs in the
same location as the target species.

\paragraph{Simulated States}\label{simulated-states}

The four possible species simulated are a grouper (\emph{Epinephelus
fuscoguttatus}, Serranidae), a shallow-reef snapper (\emph{Lutjanus
malabaricus}, Lutjanidae), a deep-reef snapper (\emph{Pristipomoides
filamentosus}, Lutjanidae), and a reef shark (\emph{Carcharhinus
amblyrhynchos}, Carcharhinidae). Two possible fleets are simulated. When
there is only one fleet, the parameters of that fleet, e.g.~selectivity
and price for a given species, are randomly drawn. When two fleets are
simulated, fleet \(\alpha\) primarily targets the grouper and shark
species, fleet \(\beta\) primarily targets the two snapper species,
though both fleets catch all four species to some extent. The set of
scenarios evaluated are shown in Table~\ref{tbl-scenes}.

\begin{longtable}[]{@{}
  >{\raggedright\arraybackslash}p{(\columnwidth - 4\tabcolsep) * \real{0.1486}}
  >{\raggedright\arraybackslash}p{(\columnwidth - 4\tabcolsep) * \real{0.5541}}
  >{\raggedright\arraybackslash}p{(\columnwidth - 4\tabcolsep) * \real{0.2973}}@{}}
\caption{Description of the three categories of simulation states
evaluated.}\label{tbl-scenes}\tabularnewline
\toprule\noalign{}
\begin{minipage}[b]{\linewidth}\raggedright
Name
\end{minipage} & \begin{minipage}[b]{\linewidth}\raggedright
Species
\end{minipage} & \begin{minipage}[b]{\linewidth}\raggedright
Fleets
\end{minipage} \\
\midrule\noalign{}
\endfirsthead
\toprule\noalign{}
\begin{minipage}[b]{\linewidth}\raggedright
Name
\end{minipage} & \begin{minipage}[b]{\linewidth}\raggedright
Species
\end{minipage} & \begin{minipage}[b]{\linewidth}\raggedright
Fleets
\end{minipage} \\
\midrule\noalign{}
\endhead
\bottomrule\noalign{}
\endlastfoot
\emph{simple} & \emph{Snapper} & \(\alpha\) \\
\emph{medium} & \emph{Snapper, Deep-Snapper, Grouper, Shark} &
\(\alpha\) \\
\emph{complex} & \emph{Snapper, Deep-Snapper, Grouper, Shark} &
\(\alpha\) and \(\beta\) \\
\end{longtable}

For each scenario set we simulated 142 states. Each state was generated
by drawing a random variable from the sets described in
Table~\ref{tbl-states}.

\begin{longtable}[]{@{}
  >{\raggedright\arraybackslash}p{(\columnwidth - 2\tabcolsep) * \real{0.2639}}
  >{\raggedright\arraybackslash}p{(\columnwidth - 2\tabcolsep) * \real{0.7361}}@{}}
\caption{Variables that define a given simulation state. See
accompanying code at
https://github.com/DanOvando/mpa-indicators-and-emulators. Clean version
will be stored and shared via Figshare once paper is
completed.}\label{tbl-states}\tabularnewline
\toprule\noalign{}
\begin{minipage}[b]{\linewidth}\raggedright
Name
\end{minipage} & \begin{minipage}[b]{\linewidth}\raggedright
Description
\end{minipage} \\
\midrule\noalign{}
\endfirsthead
\toprule\noalign{}
\begin{minipage}[b]{\linewidth}\raggedright
Name
\end{minipage} & \begin{minipage}[b]{\linewidth}\raggedright
Description
\end{minipage} \\
\midrule\noalign{}
\endhead
\bottomrule\noalign{}
\endlastfoot
KISS & Keep it Simple Stupid. When TRUE, sets habitat equal in all
patches for all species \\
MPA Response & Dictates whether effort that used to fish inside MPAs
``stays'' and is redistributed or ``leaves'' the fishery \\
habitat\_patchiness (\(\kappa\)) & Low values create smooth habitat,
higher values patchy habitat \\
Max Absolute Correlation & Scales the maximum absolute correlation
between species \\
Spatial Catchability & Whether catchability of the fishing fleets varies
in space \\
Spatial Fleet Allocation & How the fleet distributes itself in space,
one of revenue, profit per unit effort, and revenue per unit effort \\
seasonal\_movement & Whether habitat flips in one half of the year to
another \\
Spawning Aggregation & Whether spawning occurs in a concentrated
location \\
F vs.~M & Ratio of instantaneous fishing mortality rate relative to
natural mortality \\
Adult Diffusion & The diffusion rate of ``adult'' (non-larval) fish \\
Larval Diffusion & The diffusion rate of the larval fish \\
Steepness & Beverton-Holt steepness value \\
ssb0 & The unfished spawning biomass for a given species \\
Hyperallometry & Degree of hyperallometry in fecundity at age
relationship \\
Density Dependence & The timing of density dependence, one of
``global\_habitat'', ``local\_habitat'', ``pre\_dispersal'',
``post\_dispersal''. \\
ontogenetic\_shift & Whether recruitment habitat is the inverse of adult
habitat \\
Use Ports? & Whether or not to assign ports to individual fleets that
alter cost per unit effort in patches \\
Selectivity at age & Selectivity at age, set separately for every
\emph{métier} \\
Price per unit weight & Price per unit weight for every \emph{métier} \\
\end{longtable}

\paragraph{Simulating Habitats}\label{simulating-habitats}

Marine species are often distributed heterogeneously in space with
various degrees of covariance among species, i.e.~some species tend to
be found together, others tend to be found apart, and those
distributions can be clustered in space. We modified the methods
described in Thorson and Barnett (2017) to automate simulation of these
spatial habitat dynamics. The key parameters process is a parameter
\(\kappa\) that dictates the rate at which correlation among patches
decrease with distance from each other. \(\kappa\) is \textgreater0. The
closer \(\kappa\) is to 0, the greater the distance at which habitat
among patches are still correlated becomes. The larger \(\kappa\)
becomes, the less correlated habitat becomes with distance. The other
parameter is a \emph{c} by \emph{c} matrix \textbf{R}, where \emph{c} is
the number of species being simulated. The upper-triangle of \textbf{R}
is the correlation (bounded between -1 and 1) in habitat between each
simulated species. A value of 1 indicates that that given pair share
habitat perfectly, -1 that they share no habitat. When only one species
is present, \textbf{R} has no effect.

Together then \(\kappa\) dictates how ``patchy'' a simulated seascape
is, with lower values producing a smoother habitat distribution per
species, and \textbf{R} dictates how similar the habitats are across
species. Generating random draws of \(\kappa\) and \textbf{R} allows us
to automatically generate simulations that range from essentially
uniform habitats across species to cases where habitat is patchy and
individual species tend to avoid each other (@fig-hab).

\begin{figure}

\centering{

\includegraphics{results/v1.0/figs/complex_state_50_habitat.pdf}

}

\caption{\label{fig-hab}Example simulated habitat distributions. Each
panel is one species, color shows the scaled number of individuals in
each patch under unfished conditions.}

\end{figure}%

\subsubsection{Low-definition Model}\label{low-definition-model}

Much of the early MPA modeling literature used various forms of
``surplus production'' models (see Kokkalis et al. 2024) often using a
two ``patch'' system in which MPAs are modeled by tracking total biomass
in each patch while adjusting the relative size of patches and setting
fishing mortality to zero in the MPA patch (Leah R. Gerber et al. 2003).
While much more detailed models have been developed since then, these
types of models are still used, particularly in cases where the goal is
to simulate large number of fisheries or states (e.g. Sala et al. 2021).

The form of two-patch surplus production model used here is the same
Pella-Tomlinson structure used in Costello et al. (2016) although
modified to allow for the kinds of two-patch MPA dynamics used in Sala
et al. (2021). The equations are as follows, where \emph{B} is biomass,
\emph{t} is time, \(\phi\) scales the shape of the production curve
(when \$\textbackslash phi\$ = 0.188 \(B_{MSY}/K = 0.4\)), \emph{g} is
the growth rate, \emph{K} is the carrying capacity, \emph{R} is the
proportion of the total carrying capacity inside the MPA, \(F\) is the
fishing mortality rate, and \emph{m} is the movement rate:

\(F\) can be calculated with or without effort displacement. When effort
displacement is turned off, total F is constant.

\[F_t = F_{BAU}\]

When effort displacement is turned on, fishing mortality outside the
MPAs increases proportionally to the size of the protected area, in the
same manner as Sala et al. (2021) .

\[F_t = 1 - (1 - F_{BAU})^{\frac{1}{1 - R}}\]

Under ``local'' density dependence, the population model is

\[
\begin{aligned}
B_{t+1,MPA = 1} = B_{t,MPA = 1} + \frac{\phi + 1}{\phi}gB_{t,MPA = 1} \left(1 - \left(\frac{B_{t,MPA = 1}}{K \times R}\right)^{\phi} \right) \\ - m(B_{t,MPA = 1} - \frac{R}{1 - R}B_{t,MPA = 0}).
\end{aligned}
\]

\[
\begin{aligned}
B_{t+1,MPA = 0} = (1 - F_{t}) \times ( B_{t,MPA = 0} + \frac{\phi + 1}{\phi}gB_{t,MPA = 0} \left(1 - \left(\frac{B_{t,MPA = 0}}{K \times (1 - R)}\right)^{\phi} \right) \\ +  m\left(B_{t,MPA = 1} - \frac{R}{1 - R}B_{t,MPA= 0}\right) ).
\end{aligned}
\]

Under ``pooled'' density dependence, the population model is

\[B_{t,Total} = B_{t,MPA = 1} + B_{t,MPA = 0},\]

\[
\begin{aligned}
B_{t+1,MPA = 1} = B_{t,MPA = 1} + R \frac{\phi + 1}{\phi}gB_{t,Total} \left(1 - \left(\frac{B_{t,Total}}{K}\right)^{\phi} \right) \\ -  m\left(B_{t,MPA = 1} - \frac{R}{1 - R}B_{t,MPA=0}\right).
\end{aligned}
\]

\[
\begin{aligned}
B_{t+1,MPA = 0} = (1 - F_{t}) \times ( B_{t,MPA = 0} + (1 - R) \frac{\phi + 1}{\phi}gB_{t,Total} \left(1 - \left(\frac{B_{t,Total}}{K}\right)^{\phi} \right) \\ +  m\left(B_{t,MPA = 1} - \frac{R}{1 - R}B_{t,MPA=0}\right) ).
\end{aligned}
\]

The model uses ``local'' density dependence when the form of density
dependence used in the High-definition model is in some way local, and
``pooled'' when the High-definition model density dependence is in some
way global.

\paragraph{Fitting Low-definition
Model}\label{fitting-low-definition-model}

The Low-definition model has five core parameters: \(\phi\), \emph{g},
\emph{m}, \emph{K}, and \emph{D}, where \emph{D} is the depletion (B/K)
at the start of the simulation. Clearly all of these parameters could be
made up. However, the goal of this paper is to assess the impact of
model structure on simulated outcome, holding as much else constant as
possible. This means that we want to find values of these parameters
that allow the Low-definition model to best match the dynamics of the
High-definition model.

To achieve these, we estimated these five parameters along with required
variance parameters using the Stan programming language and the cmdstanR
(Gabry et al. 2024) interface. For a given scenario \emph{s} simulated
by the High-definition model, we created the following experiment.

\begin{enumerate}
\def\labelenumi{\arabic{enumi}.}
\tightlist
\item
  Fish each population from unfished biomass B0 down to a depletion of
  0.1 B0, generating a vector of total biomass \textbf{bdep} and catch
  \textbf{cdep} over time.
\item
  From that depletion of 0.1 B0, move all the fish to the left-hand side
  of the simulated seascape, such that the right 50\% of the seascape is
  empty of fish.
\item
  Close the seascape to fishing, and then let the population rebuild and
  disperse, create a vector of biomass in the left-hand \textbf{bleft}
  and right-hand \textbf{bright} side of the seascape as the population
  approaches B0 again and fills in the right-hand side of the seascape
  through adult movement and larval dispersal.
\item
  Pass \textbf{bdep,cdep,bleft, bright} to the Low-definition estimation
  model to estimate the parameters of the model.
\end{enumerate}

The concept here is that \textbf{bdep} and \textbf{cdep} provide
information on the scale of the population (B0,unfished biomass, and the
volume of catch the population can sustain), where as \textbf{bleft} and
\textbf{bright} provide additional information on \emph{g}, \(\phi\),
and \emph{m}, as the model can see how fast the population grows from a
depleted state as a function of biomass, and how fast the population
recolonizes the right-hand patch.

The structure of the model is given below, variables with \(\hat{}\)
indicate estimates by the model. Prior distributions were selected to be
generally uninformative while providing some guidance as to the general
parameter space of the variables. This model could be refined, but
provides a general means for quickly and robustly fitting the data
generated by the \emph{High-definition} model.

\[
log(bdep) \sim N(log(\hat{bdep}),\sigma_b)
\]

\[
log(cdep) \sim N(log(\hat{cdep}),\sigma_c)
\]

\[
log(bright) \sim N(log(\hat{bright}),\sigma_b)
\]

\[
log(bleft) \sim N(log(\hat{bleft}),\sigma_b)
\]

\[
log(K) \sim N(10,2)
\]

\[
m \sim N(0.2,0.2), 0 < m < 1
\]

\[
\sigma_b \sim N(0.05,0.1), \sigma_b > 0
\]

\[
D \sim N(0.5,0.5) ,0 < D < 1
\]

\[
log(phi) \sim N(log(0.188),0.4))
\]

The model was fit using penalized maximum likelihood through the
\texttt{cmdstan} \texttt{optimize} function rather than fit using full
Bayesian inference, in order to speed up computation and since we are
not interested in examining uncertainty in this analysis. Models that
failed to converge were discarded from the process (a very small
number). We confirmed that this model was capable of recovering
simulated parameters when fit to data generated by itself, and confirmed
that the model was able to match the data generated by the
High-definition model as well as possible given the presence of model
misspecification (i.e.~that the Low- and High-definition models have
fundamentally different assumptions). Once the Low-definition parameters
were estimated for a given state, the Low-definition model was then run
through the appropriate MPA scenarios.

\subsubsection{MPA Placement Scenarios}\label{mpa-placement-scenarios}

MPAs in this simulations are defined as fully no-take MPAs for all of
the species simulated in a given iteration of the model. MPA size is
defined by the proportion of the seascape placed inside the MPA. In the
High-definition case, this is the proportion of cells in the
two-dimensional space. In the Low-definition case, this is the patch
size parameter \emph{R}.

The High-definition case must also decide which patches will be used to
achieve a given MPA size. The three possible options are:

\begin{enumerate}
\def\labelenumi{\arabic{enumi}.}
\tightlist
\item
  \emph{target\_fishing}: place MPAs in descending order of catch per
  patch
\item
  \emph{avoid\_fishing}: place MPAs in ascending order of catch per
  patch
\item
  \emph{area}: start placing MPAs in bottom left corner of seascape and
  build up from there
\end{enumerate}

A given placement scenario \emph{p} then is the intersection of the
proportion of the seascape in an MPA and the strategy for placing the
MPA patches.

\subsubsection{Types of Results}\label{types-of-results}

Each simulation (a combination of scenario, state and placement)
produces simulated biomass per species and catch per species per fleet
with and without MPAs for the \emph{Low-} and \emph{High-definition}
models. We measured the effect sizes of the MPAs (MPAE) for state
\emph{s} and MPA placement scenario \emph{p} for model type \emph{m} as
percentage differences in the outcome in question

\[
MPAE_{s,p,m} = \frac{Outcome_{MPA=1,s,p,m}}{Outcome_{MPA=0,s,p,m}} -1 
\]

Where \emph{Outcome} is one of catch or biomass in the water. We then
calculate the percentage point difference between the two models as

\[
\Delta_{s,p} = MPAE_{s,p,m=Low} - MPAE_{s,p,m=High}
\]

Readers may be used to seeing differences in fishery outcomes measured
in changes in maximum sustainable yield (MSY) based reference points
such as the biomass relative to the biomass capable of producing MSY
(B/BMSY) and the fishing mortality relative to the fishing mortality
that would result in BMSY at equilibrium (F/FMSY). While these
quantities are defined for the \emph{Low-definition} model, they are
much more complex for the \emph{High-definition} model. This is because
MSY is a function both of life history and selectivity, and under the
\emph{High-definition} models selectivity can vary both among fleets and
across space. This means that the MSY-based reference points would need
to be recalculated for every equilibrium fleet distribution at a given
MPA size, and could potentially lead to confusing results such as B/BMSY
increasing even as B decreases if changes to the fleet result in BMSY
decreasing. See Kapur et al. (2021) for a discussion of these
challenges. As such, we present percentage changed based summaries of
results.

\subsection{Results}\label{results}

The results of this analysis take the form of paired predictions of the
\emph{High-definition} and \emph{Low-definition} models for a given
combination of state \emph{s}, placement strategy \emph{p}, and percent
of seascape in MPA (\emph{pmpa}). For each instance. data are generated
by the \emph{High-definition} model based on the parameters of the
\emph{state}, which are then used to estimate parameters for the
\emph{Low-definition} model. Given these parameters, a range of MPA
sizes are applied to the \emph{High-} and \emph{Low-definition} models,
with placement rules for strategy \emph{p} dictating where the MPA goes
in the \emph{High-definition model} (there is no placement option for
the Low-definition model as the two-patch dynamics do not allow it).

Possibly the clearest and most consistent prediction from the MPA
modelling literature is that more heavily fished species are more able
to benefit from MPAs, both in terms of biomass and catch. Given that, in
many of the results we analyze the role of fishing pressure in outcomes,
expressed as \(B_{bau}/B_0\) , the biomass under ``business as usual''
relative to unfished biomass. The lower this value, the more depleted
the species in question is.

\subsubsection{Example Trajectories}\label{sec-eg}

Figure~\ref{fig-eg-simple} and Figure~\ref{fig-eg-complex} illustrate
the kinds of outcomes produced by this simulation exercise.
Figure~\ref{fig-eg-simple} shows a result from the \emph{simple} set of
scenarios, in which the High-definition model has only one species and
one fleet. In this case. Figure~\ref{fig-eg-complex} shows a result from
the \emph{complex} set of scenarios, in which the High-definition model
simulations four different species deferentially targeted by two
different fishing fleets.

\begin{figure}

\centering{

\includegraphics{02_mpa_emulators_paper_files/figure-pdf/fig-eg-simple-1.pdf}

}

\caption{\label{fig-eg-simple}Example simulated effects of MPAs on
biomass and catch under the `simple' scenario (\emph{High-definition}
model only has one species and one fleet). X-axis shows the percent of
the seascape protected in a no-take MPA. Y-axis shows biomass divided
unfished biomass (A) or catch (B). Panel columns show MPA placement
strategy used in \emph{High-definition} model.}

\end{figure}%

\begin{figure}

\centering{

\includegraphics{02_mpa_emulators_paper_files/figure-pdf/fig-eg-complex-1.pdf}

}

\caption{\label{fig-eg-complex}Example simulated effects of MPAs on
biomass and catch under the `complex' scenario (\emph{High-definition}
model has four species targeted by two fleets). X-axis shows the percent
of the seascape protected in a no-take MPA. Y-axis shows biomass divided
unfished biomass (A) or catch (B). In \emph{A}, color indicates species
points are the \emph{Low-definition} model and lines are the
\emph{High-definition} model. In \emph{B}, color indicates the fleet,
and each row of the panel is a separate species. Panel columns show MPA
placement strategy used in \emph{High-definition} model.}

\end{figure}%

\subsubsection{Low-definition Compared to
High-definition}\label{sec-compare}

Each \emph{state}, \emph{placement strategy} and \emph{MPA size} is
associated with predictions of percent change in biomass and catch for
each species from the \emph{Low-} and \emph{High-definition} models
(noting that fleet-specific predictions are only possible for the
\emph{High-definition} model results). Plotting the
\emph{Low-definition} versus the \emph{High-definition} predictions
provides a measure of the degree of agreement between these two models.
Examining first changes in biomass, the \emph{Low-definition} model
predictions are almost universal positively biased, meaning that across
nearly every simulation the \emph{Low-definition} model predicted a
higher percent increase in biomass than the \emph{High-definition}
model, often substantially so. The main area of agreement between the
\emph{Low-} and High-definition models is that the potential increases
in biomass are positively correlated with the degree of fishing pressure
(as measured by BAU depletion, \(B_{bau}/B_0\)). Under the dynamics of
the \emph{Low-definition} model, it is impossible for an MPA to result
in a net decrease in biomass, whereas this result is possible under the
\emph{High-definition} model, due in the case of this model to effort
displacement from the MPA onto habitat of highly depleted species. The
complexity of the simulation state had little impact on the biomass
results, indicating that the basic presence of age structure and
two-dimensional spatial dynamics is enough to cause this discrepancy in
predicted percent changes in biomass caused by an MPA
(Figure~\ref{fig-hi-v-low}).

Examining the catch results (Figure~\ref{fig-hi-v-low}), the
\emph{Low-definition} models had a relatively clear cutoff point where
depletion values greater than 0.25 were less likely to provide positive
catch benefits. In contrast, the \emph{High-definition} models predicted
positive catch effects across many levels of depletion. This may seem
counter-intuitive, as the literature and basic biology is fairly clear
that MPA can only increase catch if the population was overfished in the
first place. The increased catches under higher depletion levels (lower
fishing pressure) predicted by the \emph{High-definition} model come
under the \emph{avoid\_fishing} placement strategies, specifically where
the fleet is concentrated in remaining fishing grounds given a constant
effort fleet model and the assumption of MPA displacement rather than
attrition. As with the biomass results, while there is a positive
correlation between the predicted catch effects of MPAs of the
\emph{Low-definition} and \emph{High-definition} models the relationship
is quite weak and generally positively biased, meaning that the
\emph{Low-definition} model on average predicted higher increases in
catch from MPAs than its paired \emph{High-definition} model. The
catch-based results are more sensitive to the complexity of the
simulation, in particular the addition of a second fleet in the
\emph{complex} scenarios, than the biomass results.

\begin{figure}

\centering{

\includegraphics{02_mpa_emulators_paper_files/figure-pdf/fig-hi-v-low-1.pdf}

}

\caption{\label{fig-hi-v-low}Percent change in biomass (top row) and
catch (bottom row) resulting from MPAs covering between 10-40\% of the
simulated seascape predicted by the High-definition (x-axis) and
Low-definition (y-axis) models. Panel columns correspond to the
simulation set used. Color indicates the biomass divided by unfished
biomass (B/B0) under business as usual (BAU) conditions without any
MPA.}

\end{figure}%

We quantified the differences between the Low- and High-definition
models by calculating \(\Delta\) as the difference in the percentage
point effect predicted by the \emph{Low-} and \emph{High-definition}
models (Figure~\ref{fig-delta}). A positive value of \(\Delta\)
indicates that the \emph{Low-definition} model predicted a higher
percent change in the given metric than the \emph{High-definition}
model, and \emph{vice versa}. The \emph{Low-definition} model
consistently overestimated the percent increase in biomass caused by a
given MPA across all levels of \(B_{bau}/B_0\) . When \(B_{bau}/B_0\)
was low, the level of positive bias followed a dome-shaped pattern,
peaking around median bias of roughly 150 percentage points between MPA
sizes of roughly 30-60\%. When \(B_{bau}/B_0\) values were higher, the
level of bias was lower at smaller MPA sizes but increased as MPA size
increased. Higher levels of \(B_{bau}/B_0\) had lower bias levels, as
the models tended to agree more that the closer a population was to
unfished biomass, the less scope for percentage increases in biomass is
possible. Predicted catch effects were less biased than predicted
biomass effects, particularly at smaller MPA sizes. The
\emph{Low-definition} model was more likely to overestimate catch
effects when \(B_{bau}/B_0\) was low, but across all levels of
\(B_{bau}/B_0\) the \emph{Low-definition} model started to become
generally negatively biased (meaning it underestimated the potential
increase in catch) as MPA size increased. However, while patterns in the
direction of bias exist, across all scenarios and levels of
\(B_{bau}/B_0\) the range of \(\Delta\) values were generally extremely
large, generally spanning 50 to 200 percentage points.

\begin{figure}

\centering{

\includegraphics{02_mpa_emulators_paper_files/figure-pdf/fig-delta-1.pdf}

}

\caption{\label{fig-delta}Distribution of percentage point difference
(\emph{Low-definition} minus \emph{High-definition}) in predicted MPA
effect size on biomass (top row) and catch (bottom row) (y-axis) as a
function of percent of seascape in MPA (x-axis). Positive y-axis values
indicates that the Low-definition predicts a more positive effect than
the High-definition model, negative a more negative effect. Columns
indicate simulation set used. Line is median value, bands are 50\% and
90\% interquantile ranges, respectively. Color indicates the biomass
divided by unfished biomass (B/B0) under business as usual (BAU)
conditions without any MPA.}

\end{figure}%

Looking across the simulations, for each \emph{state} and
\emph{placement strategy} both the \emph{Low-} and
\emph{High-definition} models provide a prediction of the maximum
percent increase in catch that could be achieved through use of MPAs.
Both models broadly agree that the potential for MPAs to increase
broadly proportional to the magnitude of fishing pressure without the
MPA. For simulations in which the fished population was heavily
depleted, the \emph{Low-definition} model consistently predicted much
greater potential increases in catch than the \emph{High-definition}
model. When the fished population was less heavily depleted, the
\emph{High-definition} model generally predicted greater potential for
increased catch than the \emph{Low-definition} model, though this was
sometimes a result of increasing fishing pressure rather than rebuilding
populations. In general, there was some positive correlation between the
maximum predicted catch potential from the \emph{Low-} and
\emph{High-definition} models, but the relationship was very loose
Figure~\ref{fig-max-catch}.

\begin{figure}

\centering{

\includegraphics{02_mpa_emulators_paper_files/figure-pdf/fig-max-catch-1.pdf}

}

\caption{\label{fig-max-catch}Maximum potential percent increase in
catch caused by an MPA predicted by the High-definition (y-axis) and
Low-definition (x-axis) models. Facets indicate simulation set.}

\end{figure}%

Rather than comparing specific predictions of percent point changes of
biomass and catch, we can also compare the ``rank'' of each fishery, in
terms of biomass or catch potential from MPAs, predicted by the
\emph{Low-} and \emph{High-definition} models. This can be useful if the
goal of the modeling exercise is not to make explicit statements about
the magnitude of changes, but simply to identify which sites might be
most amenable to achieving benefits to conservation, food security, or
both through MPAs. To that end we grouped simulations by MPA size and
then ranked the change in biomass and catch for each simulated fishery
conditional on a given MPA size. This provides a prediction of, if we
were to protect a given percentage of a seascape in an MPA, which
fisheries stand to benefit the most and the least
(Figure~\ref{fig-ranks}).

Both the \emph{Low-} and \emph{High-definition} models agreed that for
both catch and biomass, the highest rank fisheries were those with the
lowest \(B_{bau}/B_0\) values. For the \emph{simple} scenarios, there
was near 100\% agreement in the top-ranked fisheries for both biomass
and catch. However, in the \emph{complex} scenarios there was less
agreement between the models even at the top ranks. Beyond the top
tanks, while both models had similar trends, there was substantial
variation in the specific rank predicted by the two models, particularly
for catch; for the \emph{complex} scenarios, fisheries ranked in the top
75\% by the \emph{Low-definition} models could be anywhere between near
the bottom 0 or top 95\% according the \emph{High-definition} model.

\begin{figure}

\centering{

\includegraphics{02_mpa_emulators_paper_files/figure-pdf/fig-ranks-1.pdf}

}

\caption{\label{fig-ranks}Percent rank of individual fisheries
(intersection of fleet and species) conditional on proportion of
seascape protected in a no-take MPA predicted by High-definition
(x-axis) and Low-definition (y-axis). Higher percent rank indicates that
an MPA is predicted to have more positive outcomes for a given fishery,
lower worse outcomes. Top row shows results for biomass, bottom row
shows results for catch. Columns indicate simulation set. Color
indicates the biomass divided by unfished biomass (B/B0) under business
as usual (BAU) conditions without any MPA.}

\end{figure}%

The objective of an MPA modeling exercise may be even coarser than
rankings, simply asking which fisheries might benefit or lose in terms
of biomass and catch for a given MPA size. To assess this use case, we
assigned each of our simulation results to one of five quadrants:

\begin{enumerate}
\def\labelenumi{\arabic{enumi}.}
\tightlist
\item
  Biomass increases and catch decreases
\item
  Biomass increases and catch increases
\item
  Biomass decreases and catch decreases
\item
  Biomass decreases and catch increases
\item
  No effect (neither biomass or catch changes by \(\pm\) 5\%)
\end{enumerate}

This can be viewed as a ``diagnosis'' problem, where the
\emph{Low-definition} model predicts the quadrant a given MPA will place
a fishery in, which we can then compare to the ``true'' quadrant
resulting from the MPA under the \emph{High-definition} model
(Figure~\ref{fig-agreement}). Since MPA outcomes are asymptotic as MPA
size goes to 0 or 100\%, we restricted this analysis to the more
realistic range of 10-40\% of the seascape. Focusing on the
\emph{complex} set of scenarios, the \emph{Low-definition} model was a
reliable predictor of ``No Effect'' outcomes, correctly diagnosing the
\emph{High-definition} state 96\% of the time under the \emph{complex
scenario} set. However, the \emph{Low-definition} model predicted the
other quadrants quite poorly. \emph{Low-definition} predictions of
``Biomass Increases \& Catch Increases'' were correct 50\% of the time.
22\% of the cases in which the \emph{Low-definition} model predicted
``Biomass Increases \& Catch Increases'' in fact had ``No Effect'',
followed by 25\% of cases in which the true result was ``\,``Biomass
increases \& Catch Decreases''. \emph{Low-definition} predictions of
``Biomass increases \& Catch decreases'' were correct only 41\% of the
time, with ``No Effect'' being the true result in 35\% of cases. While
biomass decreases were possible according to the \emph{High-definition}
model, this outcome was not possible under the \emph{Low-definition}
model and as such the \emph{Low-definition} model did not correctly
diagnose any of those outcomes.

\begin{figure}

\centering{

\includegraphics{02_mpa_emulators_paper_files/figure-pdf/fig-agreement-1.pdf}

}

\caption{\label{fig-agreement}A) Percent agreement between quadrant
predicted by \emph{Low-definition} and \emph{High-definition} model.
Percent values normalized to sum to 100\% for each \emph{Low-definition}
prediction bin. Red dashed diagonal line shows 1:1 line which passes
through cells that indicate agreement between \emph{Low-definition} and
\emph{High-definition} models. Grey cells indicate that the
\emph{Low-definition} model made no prediction in a given quadrant.}

\end{figure}%

\subsubsection{Drivers of Differences}\label{drivers-of-differences}

The results presented here indicate that the effects of MPAs predicted
by the \emph{Low-} and \emph{High-definition} models can be
substantially different from each other, often with systemic biases in
one direction or another. We used a random forest model to predict the
magnitude of \(\Delta\) for both catch and biomass as a function of the
state variables defining the High-definition state
(Figure~\ref{fig-catch-importance},
Figure~\ref{fig-biomass-importance}). \emph{Depletion} (\(B_{bau}/B_0\))
was the most important predictor for both biomass and catch by a wide
margin. Examining the partial dependency plots for select variables,
\(\Delta\) tended to increase as \(B_{bau}/B_0\) decreases, meaning the
lower biomass was, the more the \emph{Low-definition} model
over-estimated both biomass and catch outcomes, all else being equal.
While many other variables were important predictors, none of the other
variables had as clear of a partial dependency effect as
\emph{Depletion}. This does not mean that these variables do not have an
effect on \(\Delta\), but rather that the effects of these variables may
have strong interaction effects with other variables.

\begin{figure}

\centering{

\includegraphics{02_mpa_emulators_paper_files/figure-pdf/fig-catch-importance-1.pdf}

}

\caption{\label{fig-catch-importance}A) Permutation-based variable
importance score of included covariates on predictions of model
disagreement \(\Delta\) (\emph{Low-definition} minus
\emph{High-definition}) on catch effects. B) Partial dependency plots of
selected covariates. Points are permuted values, lines are partial
dependency gradients for a given permutation set.}

\end{figure}%

\begin{figure}

\centering{

\includegraphics{02_mpa_emulators_paper_files/figure-pdf/fig-biomass-importance-1.pdf}

}

\caption{\label{fig-biomass-importance}A) Permutation-based variable
importance score of included covariates on predictions of model
disagreement (Low-definition minus High-definition) on biomass effects.
B) Partial dependency plots of selected covariates. Points are permuted
values, lines are partial dependency gradients for a given permutation
set.}

\end{figure}%

\subsubsection{Implications of Differences in
Predictions}\label{sec-imps}

The results presented so far have compared the paired predictions of the
\emph{Low-definition} and \emph{High-definition} models. As an
alternative, we can use this simulation framework to evaluate the
outcomes of following guidance based on the \emph{Low-definition} model
in a \emph{High-definition} world. To achieve this, for each state we
found the MPA size that maximized the total gain in catches according to
the \emph{Low-definition model}. For single species cases in the
\emph{Simple} scenarios, this is equal to the MPA size that maximizes
catch for the individual species. For the \emph{complex} scenario, this
is the single MPA size that maximized catch across all four species in
total. We then applied that ``food maximizing'' MPA size recommended by
the \emph{Low-definition} model to the \emph{High-definition} model and
compared the outcomes.

Under the \emph{simple} scenarios, the \emph{High-definition} model
generally agreed that MPA size proposed by the \emph{Low-definition}
model would be good for both biomass and catch, though with some
exceptions, and generally with lower catch benefits than predicted by
the Low-definition model though. However, under both the \emph{medium}
and \emph{complex} scenarios the \emph{Low-definition} prediction became
much less reliable. In particular, MPA sizes projected to have a
positive effect on catch by the \emph{Low-definition} model frequently
resulted in net reductions in catch when applied to the
\emph{High-definition} model. This difference is driven in part by the
fact that the \emph{Low-definition} model has no way of accounting for
heterogeneity in habitats and fishing fleets. For example, under the
\emph{High-definition} model, protecting 30\% of the seascape in an MPA
may protect 30\% of the habitat of one species, but near 100\% of
another species that primarily occurs in a small subset of the seascape.
Similarly, if an MPA is placed primarily on the habitat of species
\emph{A}, but effort is then displaced onto the habitat of species
\emph{B} which is not covered by the MPA, the net result can be
reduction in the population and catch of species \emph{B} (instances in
the bottom-left quadrant of Figure~\ref{fig-imps} where both biomass and
catch decrease) (Figure~\ref{fig-imps}, Figure~\ref{fig-imps2}).

\begin{figure}

\centering{

\includegraphics{02_mpa_emulators_paper_files/figure-pdf/fig-imps-1.pdf}

}

\caption{\label{fig-imps}Implications for biomass (y-axis) and catch
(x-axis) of applying the catch-maximizing MPA size predicted by the
\emph{Low-definition} model to the \emph{High-definition} model. Axes
values reflect the percent change in the given value caused by the
simulated MPA using the \emph{High-definition} model. Color reflects the
percent change in yield from applying the selected MPA size predicted by
the \emph{Low-definition} model.}

\end{figure}%

\begin{figure}

\centering{

\includegraphics{02_mpa_emulators_paper_files/figure-pdf/fig-imps2-1.pdf}

}

\caption{\label{fig-imps2}Difference in catch and biomass outcomes
predicted by the food-maximizing MPA network predicted by the
\emph{Low-definition} model (x-axis) applied to the
\emph{High-definition world} (y-axis). Top row is biomass effect, bottom
row catch, columns indicate the scenario set in question. Color
indicates the biomass divided by unfished biomass (B/B0) under business
as usual (BAU) conditions without any MPA.}

\end{figure}%

\subsection{Discussion}\label{discussion}

Models are a key tool in the ecological toolbox. Recent decades have
seen an explosion in computational capabilities that have vastly
expanded the complexity and scale of social-ecological processes that
can now be simulated, and this trend will only continue. While this
expanded world of possibilities has given us the ability to take on
previously impossible tasks, as the growing spatial and temporal scale
of the processes being modeled make the data needed to validate these
models increasingly unfeasible to collect, we can be placed in a
challenging position of depending on models that cannot be confronted
with data in the ways which we have relied on as a field in the past.

The science of MPAs provides a clear example of this challenge. While
MPAs around the world have a vast array of policy objectives, they are
increasingly being called on not only to provide protection inside their
borders but to provide positive effects to population-level outcomes
such as total biomass and food security (Sala et al. 2021). However,
with some clear exceptions, these population processes often occur on
spatial and temporal scales that severely limit effective data
collection, requiring simulation modeling.

A wide range of models have been developed to inform MPA decisions
(Fulton et al. 2015). The results of this paper show that while there
are broad areas of agreement, \emph{Low-} and \emph{High-definition}
models can provide very different predictions as to the effects of MPAs
on conservation and food security outcomes, even when set to be as
comparable as their internal dynamics allow.

The \emph{Low-} and \emph{High-definition} models agreed in some broad
ways. Both models agreed that under most conditions MPAs increased
biomass of protected species, and that MPAs could also increase fishery
catches under the right conditions, generally when the population would
have been overfished in the absence of the MPA. The predictions of the
two models were also generally correlated with each other, albeit with
often very high levels of bias and variation around this relationship
(Figure~\ref{fig-hi-v-low}). Similarly under the \emph{simple} set of
scenarios (one species, one fleet), food maximizing MPA sizes predicted
by the \emph{Low-definition} models generally at least had positive
effects on catch when applied to the \emph{High-definition} world,
though they often produced much lower catch benefits
(Figure~\ref{fig-imps2}).

While there were broad areas of agreement, the MPA effects predicted by
the \emph{High-} and \emph{Low-definition} models generally differed in
meaningful ways. In terms of biomass, the \emph{Low-definition} models
over-estimated the amount that an MPA would increase biomass in nearly
every simulated scenario, with the amount of positive bias generally
increasing with MPA size and degree of overfishing. Under the
\emph{complex} scenarios, the median bias was roughly 100 percentage
points when \(B_{bau}/B_0\) was less than 0.25 (Figure~\ref{fig-delta}).
The predicted catch effects between the \emph{Low-} and
\emph{High-definition} models were also frequently extremely different,
though with less bias than biomass. The \emph{Low-definition} model
tended to be positively biased for highly depleted stocks at lower MPA
sizes (median roughly 25 percentage points) and negatively biased for
less-depleted stocks, with increasingly negative bias as MPA size
increased (Figure~\ref{fig-delta}). From the perspective of potential
food-security implications, the \emph{Low-definition} models generally
substantially (by tens of percentage points) overestimated the maximum
potential increase in catch it might be possible to achieve from an MPA
in the \emph{High-definition} model (Figure~\ref{fig-max-catch}). When
the \emph{Low-definition} model underestimated the catch effects
generated by the \emph{High-definition} model, it was generally because
of effort displacement concentrating effort in the
\emph{High-definition} world, as opposed to rebuilding of biomass.

While the \emph{Low-definition} and \emph{High-definition} models often
made different predictions of absolute biomass and catch effects of
MPAs, their \emph{rankings} also often disagreed. This is important if a
\emph{Low-definition} model was being relied on to say prioritize
locations for MPA rather than making explicit predictions around
specific percentage point changes in biomass or catch. The two models
generally agreed that that the top-ranked fisheries in terms of
potential biomass and catch benefits of MPAs were the most depleted
fisheries. Beyond that though, while the rankings were correlated, the
two models generally ranked individual fisheries very differently,
particularly for catch (Figure~\ref{fig-ranks}). Similarly, the
Low-definition model was generally a poor predictor of the ``quadrant''
and MPA would place a fishery in in the High-definition world
(Figure~\ref{fig-agreement}). The two models almost always agreed on the
``No Effect'' quadrant, which generally occured from one or both of very
small MPA size or very high \(B_{bau}/B_0\) values. However, the
Low-definition model quadrant predictions were much less accurate when
at least one meaningful effect was predicted. For example, under the
\emph{complex} states, \emph{Low-definition} predictions of a Biomass
Increase \& Catch Increase were correct only 50\% of the time.

Interestingly, many of differences between the \emph{Low-} and
High-definition models emerged even in the \emph{simple} set of
scenarios (one species, one fleet), even in the cases where habitat of
the single species is heterogeneous. This indicates that the basic
dynamics of age-structure and two-dimensional space are enough on their
own to explain much of the discrepancy between the \emph{High-} and
\emph{Low-definition} models. In other words, we do not find evidence
that \emph{Low-definition} MPA models are consistently reliable
emulators of \emph{High-definition} MPA models so long as the world is
``simple''. Rather, even the \emph{simple} cases had very different
predictions between the two models, and these differences were only
amplified for the \emph{medium} and \emph{complex} states.

What then are the implications of different predictions between the
\emph{Low-} and \emph{High-definition} models? Under the \emph{simple}
world, MPA sizes designed to maximize food security under the
\emph{Low-definition} world generally at least provided \(\geq\) zero
benefits to biomass and catch when applied to the \emph{High-definition}
world, unless the MPAs were placed using the \emph{avoid\_fishing}
placement strategy, in which cases it was possible for both biomass and
catch benefits to be \textless{} zero. However, as the states of nature
became increasingly complex (four species and two fleets at most, which
is still far less complex than many fisheries), the performance of the
MPAs predicted to maximize food by the \emph{Low-definition} model
deteriorated, often producing net losses in catch when applied to the
\emph{High-definition} world (Figure~\ref{fig-imps2}).

From the perspective of balancing conservation and food security, the
potential consequences of these results are that MPAs designed using the
type of \emph{Low-definition} model implemented here may substantially
over-estimate or just misdiagnose the potential gains in catch an MPA
can produce, both in terms of absolute effects and rankings, if the
world in fact works in the manner stated by the \emph{High-definition}
model. From a pure conservation perspective, the potential risks of
using a Low-definition model in a High-definition world are somewhat
subtler. Under the \emph{Low-definition} model it is impossible for an
MPA to have a negative conservation effect, whereas negative effects are
possible under the \emph{High-definition} model, due primarily as
modeled here to the combined effects of displacement of fishing effort
and heterogeneous habitat. While we have no way of knowing how common
the states leading to the net conservation losses predicted by the
\emph{High-definition} model are in the real world, we do have some
evidence that they can occur (Abbott and Haynie 2012).

That being said, while there is a real risk of MPAs causing a net
conservation loss, there are clearly more ways for MPAs to benefit
conservation than to harm it. Why then does it matter if a
\emph{Low-definition} model tends to overestimate the conservation
potential of an MPA, if the objective of an MPA is not explicitly to
also benefit food security? If the goal is to rebuild a severely
overfished species and a \emph{Low-definition} model overestimates the
potential conservation gains of a given MPA size, while some rebuilding
may occur we may fall short of the actual rebuilding goal. There is also
opportunity cost; if a manager has finite capital, real or political, to
implement a conservation action, and the potential conservation benefits
of an MPA are overestimated relative to an alternative policy, a
sub-optimal choice may be made by the manager.

Our results show that the MPA effects predicted by \emph{Low-} and
\emph{High-definition} models can be meaningfully different. The general
framing of this paper has been to treat the \emph{High-definition} model
as ``reality''. This is clearly not inherently correct; just because a
model is more complex does not mean it is a better representation of
reality. We treat the \emph{High-definition} model as ``real'' here as
it is closer to the kinds of dynamics often considered to be best
practices to include in stock assessments used in fisheries management
(Punt 2023), and as the \emph{Low-definition} model is fit to the
dynamics of the \emph{High-definition} model. It is entirely possible
though that in some cases the equations used in the
\emph{Low-definition} model may be a more parsimonious representation of
reality than the \emph{High-definition} model.

We acknowledge that the simulation exercise described here is complex
and difficult to fully describe in the text of a scientific paper. To
that end, all materials needed to fully reproduce all of these results
are publicly available at
\url{https://github.com/DanOvando/mpa-indicators-and-emulators}. We
encourage readers seeking to better understand the nuances of the
simulation settings and process to examine the code along with the text
of this paper.

These results assume all data from the \emph{High-definition} model used
to fit the \emph{Low-definition} model are perfectly measured. We also
assume that all parameters (e.g.~natural mortality, growth rates,
habitat preferences), are time-invariant. These results also only
consider the simplest of economic dynamics (constant effort with or
without MPA-driven effort displacement). Other works have shown that
more complex fleet dynamics can greatly affect the outcomes of spatial
policies (Ovando et al. 2021; Cabral et al. 2019; Miller and Deacon
2016). Violations of these assumptions used in this paper will further
complicate the ability of a model to accurately predict the effects of
an MPA on conservation and catch objectives.

What we call a ``High'' resolution model should really just be
``Higher''; relative to an end-to-end ecosystem model the
\texttt{marlin} modeling framework used in the \emph{High-definition}
model is still quite simple. A paper comparing the predictions of the
\texttt{marlin} model to those of say ATLANTIS might easily show the
same level of disagreement as shown here between the \emph{High-} and
\emph{Low-definition} models. The purpose of this paper is not to
establish something like \texttt{marlin} as the ``correct'' way to model
MPAs; we instead demonstrate that the choice of model used to simulate
MPA impacts can substantially affect both the magnitude and direction of
predicted outcomes of closed areas.

To add to the complexity, much of the MPA modeling literature has
focused on simulating the impacts of ``coastal'' MPAs or more
specifically MPAs focused on what is assumed to be relatively static
habitats, often with relatively limited and diffusive movement. However,
efforts are in development to expand the use of spatial management such
as MPAs in more pelagic ecosystems, such as the international agreement
on the conservation and sustainable use of marine biodiversity of areas
beyond national jurisdiction (BBNJ,see Blasiak and Jouffray 2024). These
systems are often organized around more spatio-temporally dynamic
oceanographic features, and made up of often highly mobile species
pursued by a complex mixture of fishing fleets representing both distant
water and coastal operations. Research will be needed to consider how
best to model the impacts of MPAs in the kinds of systems considered
under the BBNJ agreement; Hampton et al. (2023) provides an example of
the many complex variables that can affect the expected outcomes of
spatial protection in pelagic systems. Further research is needed to
better establish then what models are best suited for what questions in
pelagic systems.

What then are potential solutions to this challenge? One approach is to
continue the kinds of sensitivity analyses presented here, where
alternative models are compared and checked for vastly different
predictions. This can at least provide some insight into which types of
outcomes are sensitive to what kinds of model assumptions, and what
levels of model complexity seem to at least provide consistent guidance
around outcomes of interest. A preferable approach is to actually
confront these models with data; over time as we collect more data from
across systems with MPAs before and after their implementation we may be
able to fit spatially-explicit population models (Punt 2019) to those
data, allowing use to empirically measure the predictive skill of
competing models. However, many cases are likely to simply not have the
data or capacity to estimate these kinds of models. As an alternative,
we can explore the use of more easily measured traits, such as
attributes of the size or age structure of the population in space and
time, as indicators of the potential reliability of model-based
predictions (White et al. 2011). For example, we might ask, are there
more easily measurable indices that can serve as indicators of whether
the predictions of a given model appear to be consistent with the data
that can be observed?

Countries around the world are turning increasingly to spatial
protection such as MPAs to achieve societal objectives, as embodied by
the 30x30 movement to put 30\% of the ocean in some form of spatial
protection by 2030. This means that more and more communities around the
world are going to be tasked with deciding how to design MPA networks to
achieve their objectives. This paper is not about the relative merits of
MPAs as conservation and/or fisheries management tools; that is a much
larger debate (Gaines et al. 2010; Hilborn et al. 2004). However, the
lack of clear empirical guidance (e.g.~in the form of decades of robust
empirical evidence measuring the effects of large MPA networks across
multiple objectives), means that debate and design decisions around MPAs
are often based on model outputs. These models could range from gut
feelings, heuristics such as ``protect 30\%'' , two-patch biomass
dynamics models, models with more resolved spatial and social-ecological
dynamics such as \texttt{marlin}, and all the way up to end-to-end
ecosystem models such as ATLANTIS. The results presented here show that
the choice of model used by a community can dramatically affect the
perceived potential and optimal design of MPA networks, potentially
causing at best inefficient MPA designs and at worst unintended negative
consequences for objectives such as conservation and food security.

We have reached a point in quantitative ecology where our ability to
model can far outpace our ability to monitor. The case of MPA science
illustrates this challenge; while inside-vs-outside gradients can be
measured relatively easily (Lester et al. 2009), broader
population-level effects of MPAs on conservation and fishery outcomes
can span vast areas and take years to decades to fully evolve,
constraining our ability to empirically measure their outcomes (Ovando
et al. 2021; Ferraro, Sanchirico, and Smith 2018; Nickols et al. 2019).
We must often depend on models for MPA planning without being able to
confront them with data directly. We show here that predicted effects of
MPAs on conservation and food security can be highly sensitive to the
choice of model used. While we do not yet have an answer for what model
is suitable for what question, we suggest that practitioners using
models to inform MPA science and management directly consider this
challenge and provide justification for why they believe their choice of
modeling framework is appropriate to the question at hand. While
computational efficiency clearly plays a role, we suggest that authors
should not make this the sole basis for their choice. MPAs can play an
important role in the conservation and management of marine
social-ecological systems, but we should acknowledge that our ability to
accurately predict these effects depends largely on the suitability of
modeling assumptions that may be difficult or impossible to directly
test.

\subsection{References}\label{references}

\phantomsection\label{refs}
\begin{CSLReferences}{1}{0}
\bibitem[\citeproctext]{ref-abbott2012}
Abbott, Joshua K., and Alan C. Haynie. 2012. {``What Are We Protecting?
Fisher Behavior and the Unintended Consequences of Spatial Closures as a
Fishery Management Tool.''} \emph{Ecological Applications} 22 (3):
762--77. \url{https://doi.org/10.1890/11-1319.1}.

\bibitem[\citeproctext]{ref-audzijonyte2019}
Audzijonyte, Asta, Heidi Pethybridge, Javier Porobic, Rebecca Gorton,
Isaac Kaplan, and Elizabeth A. Fulton. 2019. {``Atlantis: A Spatially
Explicit End-to-End Marine Ecosystem Model with Dynamically Integrated
Physics, Ecology and Socio-Economic Modules.''} \emph{Methods in Ecology
and Evolution} 10 (10): 1814--19.
\url{https://doi.org/10.1111/2041-210X.13272}.

\bibitem[\citeproctext]{ref-blasiak2024}
Blasiak, Robert, and Jean-Baptiste Jouffray. 2024. {``When Will the BBNJ
Agreement Deliver Results?''} \emph{Npj Ocean Sustainability} 3 (1):
1--3. \url{https://doi.org/10.1038/s44183-024-00058-6}.

\bibitem[\citeproctext]{ref-cabral2020}
Cabral, Reniel B., Darcy Bradley, Juan Mayorga, Whitney Goodell, Alan M.
Friedlander, Enric Sala, Christopher Costello, and Steven D. Gaines.
2020. {``A Global Network of Marine Protected Areas for Food.''}
\emph{Proceedings of the National Academy of Sciences} 117 (45):
28134--39. \url{https://doi.org/10.1073/pnas.2000174117}.

\bibitem[\citeproctext]{ref-cabral2019}
Cabral, Reniel B., Benjamin S. Halpern, Sarah E. Lester, Crow White,
Steven D. Gaines, and Christopher Costello. 2019. {``Designing MPAs for
Food Security in Open-Access Fisheries.''} \emph{Scientific Reports} 9
(1): 8033. \url{https://doi.org/10.1038/s41598-019-44406-w}.

\bibitem[\citeproctext]{ref-cbd2022}
CBD. 2022. {``Decision Adopted by the Conference of the Parties to the
Convention on Biological Diversity.''}

\bibitem[\citeproctext]{ref-christensen2004}
Christensen, Villy, and Carl J Walters. 2004. {``Ecopath with Ecosim:
Methods, Capabilities and Limitations.''} \emph{Ecological Modelling}
172 (2{\textendash}4): 109--39.
\url{https://doi.org/10.1016/j.ecolmodel.2003.09.003}.

\bibitem[\citeproctext]{ref-costello2016}
Costello, Christopher, Daniel Ovando, Tyler Clavelle, C. Kent Strauss,
Ray Hilborn, Michael C. Melnychuk, Trevor A. Branch, et al. 2016.
{``Global Fishery Prospects Under Contrasting Management Regimes.''}
\emph{Proceedings of the National Academy of Sciences} 113 (18):
5125--29. \url{https://doi.org/10.1073/pnas.1520420113}.

\bibitem[\citeproctext]{ref-ferraro2018}
Ferraro, Paul J., James N. Sanchirico, and Martin D. Smith. 2018.
{``Causal Inference in Coupled Human and Natural Systems.''}
\emph{Proceedings of the National Academy of Sciences}, August,
201805563. \url{https://doi.org/10.1073/pnas.1805563115}.

\bibitem[\citeproctext]{ref-fulton2015}
Fulton, Elizabeth A., Nicholas J. Bax, Rodrigo H. Bustamante, Jeffrey M.
Dambacher, Catherine Dichmont, Piers K. Dunstan, Keith R. Hayes, et al.
2015. {``Modelling Marine Protected Areas: Insights and Hurdles.''}
\emph{Phil. Trans. R. Soc. B} 370 (1681): 20140278.
\url{https://doi.org/10.1098/rstb.2014.0278}.

\bibitem[\citeproctext]{ref-cmdstanr2023}
Gabry, Jonah, Rok Češnovar, Andrew Johnson, and Steve Bronder. 2024.
\emph{Cmdstanr: R Interface to 'CmdStan'}.
\url{https://mc-stan.org/cmdstanr/}.

\bibitem[\citeproctext]{ref-gaines2010}
Gaines, Steven D., Crow White, Mark H. Carr, and Stephen R. Palumbi.
2010. {``Designing Marine Reserve Networks for Both Conservation and
Fisheries Management.''} \emph{Proceedings of the National Academy of
Sciences} 107 (43): 18286--93.
\url{https://doi.org/10.1073/pnas.0906473107}.

\bibitem[\citeproctext]{ref-gerber2003}
Gerber, Leah R., Louis W. Botsford, Alan Hastings, Hugh P. Possingham,
Steven D. Gaines, Stephen R. Palumbi, and Sandy Andelman. 2003.
{``Population Models for Marine Reserve Design: A Retrospective and
Prospective Synthesis.''} \emph{Ecological Applications} 13 (sp1):
47--64.
\url{https://doi.org/10.1890/1051-0761(2003)013\%5B0047:PMFMRD\%5D2.0.CO;2}.

\bibitem[\citeproctext]{ref-gerber2005}
Gerber, Leah R, Selina S Heppell, Ford Ballantyne, and Enric Sala. 2005.
{``The Role of Dispersal and Demography in Determining the Efficacy of
Marine Reserves.''} \emph{Canadian Journal of Fisheries and Aquatic
Sciences} 62 (4): 863--71. \url{https://doi.org/10.1139/f05-046}.

\bibitem[\citeproctext]{ref-grorud-colvert2021}
Grorud-Colvert, Kirsten, Jenna Sullivan-Stack, Callum Roberts, Vanessa
Constant, Barbara Horta e Costa, Elizabeth P. Pike, Naomi Kingston, et
al. 2021. {``The MPA Guide: A Framework to Achieve Global Goals for the
Ocean.''} \emph{Science} 373 (6560): eabf0861.
\url{https://doi.org/10.1126/science.abf0861}.

\bibitem[\citeproctext]{ref-hampton2023}
Hampton, John, Patrick Lehodey, Inna Senina, Simon Nicol, Joe Scutt
Phillips, and Kaon Tiamere. 2023. {``Limited Conservation Efficacy of
Large-Scale Marine Protected Areas for Pacific Skipjack and Bigeye
Tunas.''} \emph{Frontiers in Marine Science} 9.
\url{https://www.frontiersin.org/articles/10.3389/fmars.2022.1060943}.

\bibitem[\citeproctext]{ref-hastings1999}
Hastings, Alan, and Louis W. Botsford. 1999. {``Equivalence in Yield
from Marine Reserves and Traditional Fisheries Management.''}
\emph{Science} 284 (5419): 1537--38.
\url{https://doi.org/10.1126/science.284.5419.1537}.

\bibitem[\citeproctext]{ref-hilborn1997}
Hilborn, Ray, and Marc Mangel. 1997. \emph{The Ecological Detective:
Confronting Models with Data (MPB-28)}. Princeton University Press.
\url{https://doi.org/10.1515/9781400847310}.

\bibitem[\citeproctext]{ref-hilborn2004a}
Hilborn, Ray, Kevin Stokes, Jean-Jacques Maguire, Tony Smith, Louis W
Botsford, Marc Mangel, José Orensanz, et al. 2004. {``When Can Marine
Reserves Improve Fisheries Management?''} \emph{Ocean \& Coastal
Management} 47 (3--4): 197--205.
\url{https://doi.org/10.1016/j.ocecoaman.2004.04.001}.

\bibitem[\citeproctext]{ref-johannes1978}
Johannes, R E. 1978. {``Traditional Marine Conservation Methods in
Oceania and Their Demise.''} \emph{Annual Review of Ecology and
Systematics} 9 (1): 349--64.
\url{https://doi.org/10.1146/annurev.es.09.110178.002025}.

\bibitem[\citeproctext]{ref-kapur2021}
Kapur, M. S., M. C. Siple, M. Olmos, K. M. Privitera-Johnson, G. Adams,
J. Best, C. Castillo-Jordán, et al. 2021. {``Equilibrium Reference Point
Calculations for the Next Generation of Spatial Assessments.''}
\emph{Fisheries Research} 244 (December): 106132.
\url{https://doi.org/10.1016/j.fishres.2021.106132}.

\bibitem[\citeproctext]{ref-kokkalis2024}
Kokkalis, A., C. W. Berg, M. S. Kapur, H. Winker, N. S. Jacobsen, M. H.
Taylor, M. Ichinokawa, et al. 2024. {``Good Practices for Surplus
Production Models.''} \emph{Fisheries Research} 275 (July): 107010.
\url{https://doi.org/10.1016/j.fishres.2024.107010}.

\bibitem[\citeproctext]{ref-lester2009}
Lester, S. E., B. S. Halpern, K. Grorud-Colvert, J. Lubchenco, B. I.
Ruttenberg, S. D. Gaines, S. Airamé, and R. R. Warner. 2009.
{``Biological Effects Within No-Take Marine Reserves: A Global
Synthesis.''} \emph{Marine Ecology Progress Series} 384: 3346.

\bibitem[\citeproctext]{ref-lorenzen2022}
Lorenzen, Kai. 2022. {``Size- and Age-Dependent Natural Mortality in
Fish Populations: Biology, Models, Implications, and a Generalized
Length-Inverse Mortality Paradigm.''} \emph{Fisheries Research} 255
(November): 106454. \url{https://doi.org/10.1016/j.fishres.2022.106454}.

\bibitem[\citeproctext]{ref-miller2016a}
Miller, Steve J., and Robert T. Deacon. 2016. {``Protecting Marine
Ecosystems: Regulation Versus Market Incentives.''} \emph{Marine
Resource Economics} 32 (1): 83--107.
\url{https://doi.org/10.1086/689214}.

\bibitem[\citeproctext]{ref-murdoch2002}
Murdoch, W. W., B. E. Kendall, R. M. Nisbet, C. J. Briggs, E. McCauley,
and R. Bolser. 2002. {``Single-Species Models for Many-Species Food
Webs.''} \emph{Nature} 417 (6888): 541--43.
\url{https://doi.org/10.1038/417541a}.

\bibitem[\citeproctext]{ref-nickols2019}
Nickols, Kerry J., J. Wilson White, Dan Malone, Mark H. Carr, Richard M.
Starr, Marissa L. Baskett, Alan Hastings, and Louis W. Botsford. 2019.
{``Setting Ecological Expectations for Adaptive Management of Marine
Protected Areas.''} \emph{Journal of Applied Ecology} 56 (10): 2376--85.
https://doi.org/\url{https://doi.org/10.1111/1365-2664.13463}.

\bibitem[\citeproctext]{ref-oleary2016}
O'Leary, Bethan C., Marit Winther-Janson, John M. Bainbridge, Jemma
Aitken, Julie P. Hawkins, and Callum M. Roberts. 2016. {``Effective
Coverage Targets for Ocean Protection.''} \emph{Conservation Letters} 9
(6): 398--404. https://doi.org/\url{https://doi.org/10.1111/conl.12247}.

\bibitem[\citeproctext]{ref-ovando2023a}
Ovando, Daniel, Darcy Bradley, Echelle Burns, Lennon Thomas, and James
Thorson. 2023. {``Simulating Benefits, Costs and Trade-Offs of Spatial
Management in Marine Social-Ecological Systems.''} \emph{Fish and
Fisheries} n/a (n/a). \url{https://doi.org/10.1111/faf.12804}.

\bibitem[\citeproctext]{ref-ovando2021}
Ovando, Daniel, Jennifer E. Caselle, Christopher Costello, Olivier
Deschenes, Steven D. Gaines, Ray Hilborn, and Owen Liu. 2021.
{``Assessing the Population{-}Level Conservation Effects of Marine
Protected Areas.''} \emph{Conservation Biology} 35 (6): 1861--70.
\url{https://doi.org/10.1111/cobi.13782}.

\bibitem[\citeproctext]{ref-ovando2023}
Ovando, Daniel, Owen Liu, Renato Molina, Ana Parma, and Cody Szuwalski.
2023. {``Global Effects of Marine Protected Areas on Food Security Are
Unknown.''} \emph{Nature} 621 (7979): E34--36.
\url{https://doi.org/10.1038/s41586-023-06493-8}.

\bibitem[\citeproctext]{ref-pella1969}
Pella, Jerome J., and Patrick K. Tomlinson. 1969. {``A Generalized Stock
Production Model.''} \emph{Inter-American Tropical Tuna Commission
Bulletin} 13 (3): 416--97. \url{http://aquaticcommons.org/3536/}.

\bibitem[\citeproctext]{ref-pelletier2005}
Pelletier, Dominique, and Stéphanie Mahévas. 2005. {``Spatially Explicit
Fisheries Simulation Models for Policy Evaluation.''} \emph{Fish and
Fisheries} 6 (4): 307--49.
\url{https://doi.org/10.1111/j.1467-2979.2005.00199.x}.

\bibitem[\citeproctext]{ref-punt2019}
Punt, André E. 2019. {``Spatial Stock Assessment Methods: A Viewpoint on
Current Issues and Assumptions.''} \emph{Fisheries Research} 213 (May):
132--43. \url{https://doi.org/10.1016/j.fishres.2019.01.014}.

\bibitem[\citeproctext]{ref-punt2023}
---------. 2023. {``Those Who Fail to Learn from History Are Condemned
to Repeat It: A Perspective on Current Stock Assessment Good Practices
and the Consequences of Not Following Them.''} \emph{Fisheries Research}
261 (May): 106642. \url{https://doi.org/10.1016/j.fishres.2023.106642}.

\bibitem[\citeproctext]{ref-sala2021}
Sala, Enric, Juan Mayorga, Darcy Bradley, Reniel B. Cabral, Trisha B.
Atwood, Arnaud Auber, William Cheung, et al. 2021. {``Protecting the
Global Ocean for Biodiversity, Food and Climate.''} \emph{Nature},
March, 1--6. \url{https://doi.org/10.1038/s41586-021-03371-z}.

\bibitem[\citeproctext]{ref-sutherland2023}
Sutherland, Chris, Darragh Hare, Paul J. Johnson, Daniel W. Linden,
Robert A. Montgomery, and Egil Droge. 2023. {``Practical Advice on
Variable Selection and Reporting Using Akaike Information Criterion.''}
\emph{Proceedings of the Royal Society B: Biological Sciences} 290
(2007): 20231261. \url{https://doi.org/10.1098/rspb.2023.1261}.

\bibitem[\citeproctext]{ref-thorson2020}
Thorson, James T. 2020. {``Predicting Recruitment Density Dependence and
Intrinsic Growth Rate for All Fishes Worldwide Using a Data-Integrated
Life-History Model.''} \emph{Fish and Fisheries} 21 (2): 237--51.
\url{https://doi.org/10.1111/faf.12427}.

\bibitem[\citeproctext]{ref-thorson2021a}
Thorson, James T., Steven J. Barbeaux, Daniel R. Goethel, Kelly A.
Kearney, Edward A. Laman, Julie K. Nielsen, Matthew R. Siskey, Kevin
Siwicke, and Grant G. Thompson. 2021. {``Estimating Fine-Scale Movement
Rates and Habitat Preferences Using Multiple Data Sources.''} \emph{Fish
and Fisheries} 22 (6): 1359--76.
\url{https://doi.org/10.1111/faf.12592}.

\bibitem[\citeproctext]{ref-thorson2017a}
Thorson, James T., and Lewis A. K. Barnett. 2017. {``Comparing Estimates
of Abundance Trends and Distribution Shifts Using Single- and
Multispecies Models of Fishes and Biogenic Habitat.''} \emph{ICES
Journal of Marine Science} 74 (5): 1311--21.
\url{https://doi.org/10.1093/icesjms/fsw193}.

\bibitem[\citeproctext]{ref-white2011}
White, J. Wilson, Louis W. Botsford, Marissa L. Baskett, Lewis AK
Barnett, R. Jeffrey Barr, and Alan Hastings. 2011. {``Linking Models
with Monitoring Data for Assessing Performance of No-Take Marine
Reserves.''} \emph{Frontiers in Ecology and the Environment} 9 (7):
390--99. \url{https://doi.org/10.1890/100138}.

\end{CSLReferences}



\end{document}
